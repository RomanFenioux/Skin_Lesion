\documentclass[a4paper,10pt]{article}
\usepackage[utf8]{inputenc}
\usepackage{xargs}                      % Use more than one optional parameter in a new commands
\usepackage[pdftex,dvipsnames]{xcolor}  % Coloured text etc.
%\usepackage{ansmath}
\usepackage[colorinlistoftodos,prependcaption,textsize=tiny]{todonotes}
% note: disable any of the folling commands by adding 'disable' as a first option after \todo
% see example with the third one
\newcommandx{\tobecompleted}[2][1=]{\todo[linecolor=red,backgroundcolor=red!25,bordercolor=red,#1]{#2}}
\newcommandx{\thiswillnotshow}[2][1=]{\todo[disable,linecolor=OliveGreen,backgroundcolor=OliveGreen!25,bordercolor=OliveGreen,#1]{#2}}

%\DeclareMathOperator*{\argmin}{argmin} 

%opening
\title{PRIM Project\\ Segmentation of skin lesions}
\author{Roman Fenioux}

\begin{document}
\maketitle
\newpage
\begin{abstract}
In this project I will compare different segmentation methods for skin lesions. \tobecompleted{intro about skin lesion, importance, etc...}
\end{abstract}

\section{Segmentation}
\subsection{Thresholding}
\paragraph{}
The easiest method that comes to mind is thresholding since the skin lesions are usually darker than the surrounding skin. Given a grayscale image, we can classify pixels as being part of the region of interest (ROI) or of the background based on their intensities.
\paragraph{Otsu~\cite{Otsu1979}}\tobecompleted{Otsu algorithm description}
Our objective is to find a good threshold to separate the lesion from the surrounding skin. Otsu's approach consist seeing the histogram as a probability distribution, and then choosing the threshold that maximizes the separability measure $\eta$ or equivalently the variance $\sigma_B^2$ between the two classes obtained, as defined in equation \ref{eq:otsuvariance}.

\begin{equation} \label{eq:otsuvariance}
  \sigma_B^2 = \omega_0 (\mu_0 - \mu_T)^2 + \omega_1 (\mu_1 - \mu_T)^2 
\end{equation}
\begin{equation} \label{eq:optimthresh}
  k^* = arg\max_{1<k<L} \sigma_B^2(k)   
\end{equation}
\tobecompleted{correct the argmax typo, with package ansmath and DeclareMathOperator }



\section{Pre-processing}
\subsection{Hair removal}
\paragraph{Dullrazor~\cite{Dullrazor1997}} algorithm is the state-of-the-art method for hair removal. It consists in the following steps:
\begin{enumerate}
 \item Locating dark hair with a grayscale morphological closing operation with vertical, horizontal and diagonals structure elements on the three RGB channels. We obtain the hair mask by thresholding the absolute difference with the original image.
 \item Denoising the hair mask and interpolating the hair pixels with neighbor non-hair pixels. 
 \item Removing the remaining hair artefacts with an adapted median filter, only applied on the pixels located in the enlarged hair regions
 (morphological dilatation of the hair mask).
\end{enumerate}


\subsection{Color space}
The segmentation performance is influence by the colorspace used. Some methods \tobecompleted{cite them} simply use the blue channel from RGB colorspace, but Garnavi, Celebi and al~\cite{Garnavi2010} have shown that, at least for a threshold-based approach, the X channel from CIE-XYZ color space gives better performances.

\bibliographystyle{plain}
\bibliography{PRIM-rapport}

\end{document}
